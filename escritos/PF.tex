\documentclass[brazil]{abnt}
\usepackage[utf8]{inputenc}
\usepackage[brazil]{babel}
%opening

\makeatletter
\usepackage{babel}
\makeatother
\begin{document}

\autor{Renato dos Santos Cerqueira}
\titulo{Title Placeholder}
\orientador{Adriano Joaquim de Oliveira Cruz}
\comentario{Monografia apresentada para obtenção do Grau de Bacharel em Ciência da Computação pela Universidade 
Federal do Rio de Janeiro.}
\instituicao{Departamento de Ciência da Computação \par Instituto de Matemática \par Universidade Federal do Rio de Janeiro}
\local{Rio de Janeiro - RJ, Brasil}
\data{21/12/2012}

\capa
\folhaderosto

\begin{folhadeaprovacao}
Monografia de Projeto Final de Graduação sob o título \textit{``\ABNTtitulodata''},
defendida por \ABNTautordata~e aprovada em \ABNTdatadata, no Rio de Janeiro,
Estado do Rio de Janeiro, pela banca examinadora constituída pelos
professores: \setlength{\ABNTsignthickness}{0.4pt}

\assinatura{Prof. Ph.D. Adriano Joaquim de Oliveira Cruz\\ Orientador} \assinatura{???\\ Universidade ???} \assinatura{???\\ Universidade ???}
\end{folhadeaprovacao}

\begin{resumo}
O objetivo deste trabalho é fazer uma \textit{engine} para o \textit{videogame Nintendo DS\texttrademark} e também um editor de fases 
e configurações, como seria feito numa equipe de desenvolvimento de um jogo grande, dando a possibilidade aos \textit{designers} 
de fazerem seus \textit{sprites} e criarem as fases com eles, sem que fosse necessário mexer com códigos.
\end{resumo}

\begin{abstract}
The objective of this paper is to make an engine to the Nintendo DS\texttrademark system and a level and configurations editor, as it
would be done in a development team in a big game, giving designers the possibility to make their sprites and create their levels 
without touching actual source code.
\end{abstract}

\chapter*{Dedicatória}

\chapter*{Agradecimentos}

\tableofcontents{}
\listoffigures
\listoftables

\chapter{Introdução\label{cap:introducao}}

\vfill{}
\begin{flushright}{}``\emph{Não há fé inabalável senão aquela que}\\
\emph{pode encarar a razão face a face, em}\\
\emph{todas as épocas da Humanidade.}''\\
{\small Allan Kardec}\end{flushright}{\small \par}
\vfill{}

Neste capítulo são apresentados o objetivo desta monografia e a estrutura
da mesma.
\newpage


\section{Objetivo deste trabalho}

É comum ver retrospectivas lembrando da evolução dos jogos ao longo dos últimos anos. Jogos eletrônicos, como é sabido, são coisas relativamente recentes, sendo o primeiro o primeiro exemplo conhecido de 1947, quando Thomas T. Goldsmith Jr. e Estle Ray Mann introduziram sua patente para um Dispositivo de Entretenimento usando Tubo de Raios Catódicos.

No entanto, muita coisa mudou nas poucas décadas da existências dos jogos eletrônicos. Quando vemos jogos produzidos na época do Atari 2600 e Magnavox Odyssey, estes eram criações de um único desenvolvedor, muitas vezes em períodos curtos de tempo, sem nem ao menos identificar o time de desenvolvimento. Era comum a aparição de \textit{Easter Eggs} que os desenvolvedores usavam para tentar identificar a sua criação.

Hoje em dia, porém, a indústria de jogos sendo a grande indústria que é, conta com superproduções com uma quantidade quase infindável de desenvolvedores, designers, criadores de fases, dentre outros tantos profissionais trabalhando cada um com sua especialidade.

Neste trabalho o que tentamos fazer é nos inserir um pouco no contexto desses times de criação modernos, vendo apenas uma pequena parte dessa tão grande cadeia de trabalho: a criação de ferramentas para facilitar a interação entre equipes de ramos diferentes. Mais especificamente, tentamos criar um jogo em duas dimensões, do tipo plataforma para o videogame portátil Nintendo DS. E tentamos fazer com que a engine que roda o jogo seja a mais fácil de reutilizar através de ferramentas de criação de fases e de configuração do personagem, itens, dentre outras coisas que compõe o mundo do jogo.

Nosso objetivo, porém, é fazer o papel do desenvolvedor. E somente criar as ferramentas, imaginando como os outros times as usariam.

\section{Estrutura da monografia}

No capítulo~\ref{cap:engine} é apresentado o estado da arte
na área de segurança de redes ...


\chapter{A criação do jogo\label{cap:engine}}

\vfill{}
\begin{flushright}{}``\emph{Navegar é preciso, viver não.}''\\
{\small Luís de Camões}\end{flushright}{\small \par}
\vfill{}

Neste capítulo é apresentado o desenvolvimento da parte para o portátil Nintendo DS.
\newpage


\section{Introdução}

O que é exatamente um jogo de plataforma em duas dimensões? Podemos pensar em exemplos clássicos, como Super Mario World para o Super Nintendo, ou Sonic The Hedgehog para o Sega Mega Drive.

Nem todos os jogos plataforma são em duas dimensões. Mas vamos aqui nos focar num jogo plataforma em duas dimensões para facilitar o entendimento e o desenvolvimento de soluções. Nessa modalidade, uma característica comum presente em muitos jogos é o uso de \textit{sprites}, figuras de tamanho pré-definido, que compõem todo o mundo, como um grande quebra-cabeças. Nesse tipo de jogo, todos os objetos costumam ser feitos da combinações de sprites e muitas vezes um sprite é usado mais de uma vez em diferentes posições.

Podemos definir um jogo de plataforma por suas características comuns: o jogador controla um personagem que se movimenta para os lados, pode ou não saltar, ter ou não algum tipo de arma e em geral o objetivo é percorrer uma série de fases derrotando pequenos inimigos, para ao final derrotar algum grande inimigo e recuperar alguma coisa. Claro que essa é uma definição bem aberta, e é nela que vamos tentar nos inspirar ao escrever a parte que será executada no Nintendo DS.

Nosso objetivo é criar uma engine que leia algum tipo de arquivo de configuração, onde estarão detalhadas informações sobre quais são as imagens que compõem o personagem principal (ou personagens, no caso de haver uma escolha); informações sobre os inimigos como por exemplo a qual tipo de movimento ele obedece; descrição dos itens, como por exemplo qual efeito ele causa no personagem principal ou nos inimigos e informações sobre as fases, posicionamento de itens, inimigos e as imagens que as compõem.

\chapter{Conclusões e trabalhos futuros\label{cap:conclusao}}

\vfill{}
\begin{flushright}{}``\emph{Nada se cria, nada se perde, tudo se
transforma.}''\\
{\small Lavousier}\end{flushright}{\small \par}
\vfill{}

Neste capítulo é apresentado as conclusões e alguns trabalhos futuros
...
\newpage


\section{Conclusões}

{\bf Alguns itens interessantes para a conclusão de um projeto de graduação}

Qual foi o resultado do seu trabalho? melhora na área, testes positivos ou negativos?
Você acha que o mecanismo gerado produziu resultados interessantes?
Quais os problemas que você encontrou na elaboração do projeto?
E na implementação do protótipo?
Que conclusão você tirou das ferramentas utilizadas? (heurísticas, prolog, ALE, banco de dados).
Em que outras áreas você julga que este trabalho seria interessante de ser aplicado?
Que tipo de continuidade você daria a este trabalho?
Que tipo de conhecimento foi necessário para este projeto de graduação?
Para que serviu este trabalho na sua formação?


\bibliographystyle{abnt-alf}
\bibliography{PF}


\anexo


\chapter{Ferramentas utilizadas}

Foi feita uma análise de algumas ferramentas que são muito usadas
por atacantes (hackers) para a confecção de ataques. Estas ferramentas
são muito úteis em vários aspectos, tais como: (1) o levantamento
de informações sobre o alvo, (2) que tipo de serviços estão disponíveis
no alvo, (3) quais as possíveis vulnerabilidades do alvo, entre outras
informações. As ferramentas analisadas foram o \emph{nmap}~\cite{dns2002},
o \emph{nessus}~\cite{William86}, o \emph{saint}~\cite{rhoden}, além
de alguns comandos de sistemas operacionais (UNIX-Like e Windows-Like)
usados para rede, tais como o \emph{ping, nslookup e whois}. 
\end{document}
